\documentclass[11pt]{article}
\usepackage[margin=1in]{geometry} 
\usepackage{amsmath,amsthm,amssymb}
\usepackage{float}
\usepackage{bbm}
% MacrosObjective (LaTeX) Jan 28, 2004

% FONT DEFINITIONS
\def\reduce#1{{\small  #1}}
\def\c#1{\mathcal #1}
\def\foot#1{\footnote{#1}}

\newcommand{\veps}{\varepsilon}

% Bold face definitions
\newcommand{\nmathbf}{\bm}
\def\b#1{{\bf #1}}

\def\bfA{\nmathbf A}
\def\bfB{\nmathbf B}
\def\bfC{\nmathbf C}
\def\bfD{\nmathbf D}
\def\bfE{\nmathbf E}
\def\bfF{\nmathbf F}
\def\bfG{\nmathbf G}
\def\bfH{\nmathbf H}
\def\bfI{\nmathbf I}
\def\bfJ{\nmathbf J}
\def\bfK{\nmathbf K}
\def\bfL{\nmathbf L}
\def\bfM{\nmathbf M}
\def\bfN{\nmathbf N}
\def\bfO{\nmathbf O}
\def\bfP{\nmathbf P}
\def\bfQ{\nmathbf Q}
\def\bfR{\nmathbf R}
\def\bfS{\nmathbf S}
\def\bfT{\nmathbf T}
\def\bfU{\nmathbf U}
\def\bfV{\nmathbf V}
\def\bfW{\nmathbf W}
\def\bfX{\nmathbf X}
\def\bfY{\nmathbf Y}
\def\bfZ{\nmathbf Z}

\def\bfa{\nmathbf a}
\def\bfb{\nmathbf b}
\def\bfc{\nmathbf c}
\def\bfd{\nmathbf d}
\def\bfe{\nmathbf e}
\def\bff{\nmathbf f}
\def\bfg{\nmathbf g}
\def\bfh{\nmathbf h}
\def\bfi{\nmathbf i}
\def\bfj{\nmathbf j}
\def\bfk{\nmathbf k}
\def\bfl{\nmathbf l}
\def\bfm{\nmathbf m}
\def\bfn{\nmathbf n}
\def\bfo{\nmathbf o}
\def\bfp{\nmathbf p}
\def\bfq{\nmathbf q}
\def\bfr{\nmathbf r}
\def\bfs{\nmathbf s}
\def\bft{\nmathbf t}
\def\bfu{\nmathbf u}
\def\bfv{\nmathbf v}
\def\bfw{\nmathbf w}
\def\bfx{\nmathbf x}
\def\bfy{\nmathbf y}
\def\bfz{\nmathbf z}

\def\bfalpha  {\nmathbf \alpha}
\def\bfbeta   {\nmathbf \beta}
\def\bfgamma  {\nmathbf \gamma}
\def\bfdelta  {\nmathbf \delta}
\def\bfepsilon{\nmathbf \epsilon}
\def\bfzeta   {\nmathbf \zeta}
\def\bfeta    {\nmathbf \eta}
\def\bftheta  {\nmathbf \theta}
\def\bfiota   {\nmathbf \iota}
\def\bfkappa  {\nmathbf \kappa}
\def\bflambda {\nmathbf \lambda}
\def\bfmu     {\nmathbf \mu}
\def\bfnu     {\nmathbf \nu}
\def\bfxi     {\nmathbf \xi}
\def\bfomicron{\nmathbf \omicron}
\def\bfpi     {\nmathbf \pi}
\def\bfrho    {\nmathbf \rho}
\def\bfsigma  {\nmathbf \sigma}
\def\bftau    {\nmathbf \tau}
\def\bfupsilon{\nmathbf \upsilon}
\def\bfphi    {\nmathbf \phi}
\def\bfpsi    {\nmathbf \psi}
\def\bfchi    {\nmathbf \chi}
\def\bfomega  {\nmathbf \omega}
\def\bfvarepsilon  {\nmathbf \varepsilon}

\def\bfAlpha  {\nmathbf \Alpha}
\def\bfBeta   {\nmathbf \Beta}
\def\bfGamma  {\nmathbf \Gamma}
\def\bfDelta  {\nmathbf \Delta}
\def\bfEpsilon{\nmathbf \Epsilon}
\def\bfZeta   {\nmathbf \Zeta}
\def\bfEta    {\nmathbf \Eta}
\def\bfTheta  {\nmathbf \Theta}
\def\bfIota   {\nmathbf \Iota}
\def\bfKappa  {\nmathbf \Kappa}
\def\bfLambda {\nmathbf \Lambda}
\def\bfMu     {\nmathbf \Mu}
\def\bfNu     {\nmathbf \Nu}
\def\bfXi     {\nmathbf \Xi}
\def\bfOmicron{\nmathbf \Omicron}
\def\bfPi     {\nmathbf \Pi}
\def\bfPsi     {\nmathbf \Psi}
\def\bfRho    {\nmathbf \Rho}
\def\bfSigma  {\nmathbf \Sigma}
\def\bfTau    {\nmathbf \Tau}
\def\bfUpsilon{\nmathbf \Upsilon}
\def\bfPhi    {\nmathbf \Phi}
\def\bfChi    {\nmathbf \Chi}
\def\bfOmega  {\nmathbf \Omega}

\newcommand{\ttheta}{\tilde{\theta}}
\newcommand{\bfzero}{{\nmathbf 0}}
\newcommand{\bfone}{{\nmathbf 1}}
\newcommand{\vareps}{\varepsilon}
\def\bfvareps{\nmathbf \varepsilon}
\newcommand{\tgamma}{\tilde\gamma}

%% Calligraphic
\newcommand{\cfA}{\c{A}}
\newcommand{\cfB}{\c{B}}
\newcommand{\cfC}{\c{C}}
\newcommand{\cfD}{\c{D}}
\newcommand{\cfE}{\c{E}}
\newcommand{\cfF}{\c{F}}
\newcommand{\cfG}{\c{G}}
\newcommand{\cfH}{\c{H}}
\newcommand{\cfI}{\c{I}}
\newcommand{\cfJ}{\c{J}}
\newcommand{\cfK}{\c{K}}
\newcommand{\cfL}{\c{L}}
\newcommand{\cfM}{\c{M}}
\newcommand{\cfN}{\c{N}}
\newcommand{\cfO}{\c{O}}
\newcommand{\cfP}{\c{P}}
\newcommand{\cfQ}{\c{Q}}
\newcommand{\cfR}{\c{R}}
\newcommand{\cfS}{\c{S}}
\newcommand{\cfT}{\c{T}}
\newcommand{\cfU}{\c{U}}
\newcommand{\cfV}{\c{V}}
%\newcommand{\cfX}{\cal X}
%\newcommand{\cfX}{\symbol{\mathcal{X}}}
\newcommand{\cfX}{\c{X}}
\newcommand{\cfY}{\c{Y}}
\newcommand{\cfZ}{\c{Z}}

\def\boldfacefake#1{\kern-4pt
    \hbox{ \mathsurround=0pt
    \hbox to 0.4pt{$#1$\hss}\hbox to 0.4pt{$#1$\hss}\hbox {$#1$}}}

\newcommand{\bcfA}{\boldsymbol{\mathcal{A}}}
\newcommand{\bcfB}{\boldsymbol{\mathcal{B}}}
\newcommand{\bcfC}{\boldsymbol{\mathcal{C}}}
\newcommand{\bcfD}{\boldsymbol{\mathcal{D}}}
\newcommand{\bcfE}{\boldsymbol{\mathcal{E}}}
\newcommand{\bcfF}{\boldsymbol{\mathcal{F}}}
\newcommand{\bcfG}{\boldsymbol{\mathcal{G}}}
\newcommand{\bcfH}{\boldsymbol{\mathcal{H}}}
\newcommand{\bcfI}{\boldsymbol{\mathcal{I}}}
\newcommand{\bcfJ}{\boldsymbol{\mathcal{J}}}
\newcommand{\bcfK}{\boldsymbol{\mathcal{K}}}
\newcommand{\bcfL}{\boldsymbol{\mathcal{L}}}
\newcommand{\bcfM}{\boldsymbol{\mathcal{M}}}
\newcommand{\bcfN}{\boldsymbol{\mathcal{N}}}
\newcommand{\bcfO}{\boldsymbol{\mathcal{O}}}
\newcommand{\bcfP}{\boldsymbol{\mathcal{P}}}
\newcommand{\bcfQ}{\boldsymbol{\mathcal{Q}}}
\newcommand{\bcfR}{\boldsymbol{\mathcal{R}}}
\newcommand{\bcfS}{\boldsymbol{\mathcal{S}}}
\newcommand{\bcfT}{\boldsymbol{\mathcal{T}}}
\newcommand{\bcfU}{\boldsymbol{\mathcal{U}}}
\newcommand{\bcfV}{\boldsymbol{\mathcal{V}}}
\newcommand{\bcfW}{\boldsymbol{\mathcal{W}}}
\newcommand{\bcfX}{\boldsymbol{\mathcal{X}}}
\newcommand{\bcfY}{\boldsymbol{\mathcal{Y}}}
\newcommand{\bcfZ}{\boldsymbol{\mathcal{Z}}}

% MATHEMATICAL NOTATION

% Operators

\newcommand{\g}{\,\vert\,}
\newcommand{\bg}{\;\bigg\vert\;}
\newcommand{\p}{\mbox{P}}
\newcommand{\D}{\mbox{D}}
\newcommand{\E}{\mbox{E}}
\newcommand{\Mo}{\mbox{Mo}}
\newcommand{\Me}{\mbox{Me}}
\newcommand{\Cov}{\mbox{Cov}}
\newcommand{\Var}{\mbox{Var}}
\newcommand{\Corr}{\mbox{Corr}}
\newcommand{\Q}{\mbox{Q}}

% Distributions
\newcommand{\Bb}{\mbox{Bb}}
\newcommand{\Be}{\mbox{Be}}
\newcommand{\Bi}{\mbox{Bi}}
\newcommand{\Br}{\mbox{Br}}
\newcommand{\Ca}{\mbox{Ca}}
\newcommand{\Di}{\mbox{Di}}
\newcommand{\Ex}{\mbox{Ex}}
\newcommand{\Fs}{\mbox{Fs}}
\newcommand{\Ga}{\mbox{Ga}}
\newcommand{\Ge}{\mbox{Ge}}
\newcommand{\Gg}{\mbox{Gg}}
\newcommand{\Hy}{\mbox{Hy}}
\newcommand{\Ig}{\mbox{Ig}}
\newcommand{\Ip}{\mbox{Ip}}
\newcommand{\Lo}{\mbox{Lo}}
\newcommand{\Mu}{\mbox{Mu}}
\newcommand{\N}{\mbox{N}}
\newcommand{\Nb}{\mbox{Nb}}
\newcommand{\Ng}{\mbox{Ng}}
\newcommand{\Nw}{\mbox{Nw}}
\newcommand{\Pa}{\mbox{Pa}}
\newcommand{\Po}{\mbox{Po}}
\newcommand{\Pg}{\mbox{Pg}}
\newcommand{\Pn}{\mbox{Pn}}
\newcommand{\Ra}{\mbox{Ra}}
\newcommand{\St}{\mbox{St}}
\newcommand{\Un}{\mbox{Un}}
\newcommand{\Wi}{\mbox{Wi}}

% General Mathematics
\newcommand{\dd}[1]{\,d#1}
\newcommand{\barx}{\mbox{$\overline x$}}
%\newcommand{\comb}[2]{{#1\choose#2}}
\newcommand{\ontop}[2]{{#1\atop#2}}
\newcommand{\h}{\hbox{{\small$1\over2$}}}
\newcommand{\ok}{\hfill\fbox{}}

% Abbreviations
\newcommand{\brow}[2]{\mbox{$\{{#1}_1,\ldots,{#1}_{#2}\}$}}
\newcommand{\prow}[2]{\mbox{$({#1}_1,\ldots,{#1}_{#2})$}}
\newcommand{\row}[2]{\mbox{${#1}_1,\ldots,{#1}_{#2}$}}
\newcommand{\data}{\row{x}{n}}
\newcommand{\bdata}{\prow{x}{n}}
\newcommand{\ie}{\emph{i.e.},\ }
\newcommand{\co}{\emph{cf.}\ }
\newcommand{\eg}{\emph{e.g.}, }
\newcommand{\etalc}{\emph{et al.},\ }
\newcommand{\etal}{\emph{et al.}\ }

\newcommand{\bpro}{\begin{proof}}
\newcommand{\epro}{\end{proof}}
\newcommand{\be}{\begin{eqnarray}}
\newcommand{\ee}{\end{eqnarray}}
\newcommand{\ba}{\begin{eqnarray*}}
\newcommand{\ea}{\end{eqnarray*}}
\newcommand{\bc}{\begin{center}}
\newcommand{\ec}{\end{center}}
\newcommand{\btab}[1]{\begin{tabular}{#1}}
\newcommand{\etab}{\end{tabular}}

\newcommand{\go}{\rightarrow}
\newcommand{\gor}{\rightarrow}
\newcommand{\goi}{\rightarrow \infty}
\newcommand{\ul}{\underline}
\newcommand{\ol}{\overline}
\newcommand{\fr}{\frac}
\newcommand{\pn}{\par\noindent}
\newcommand{\nc}{\nonumber\\}
\newcommand{\ssum}{\mbox{$\sum$}}
\newcommand{\hhline}{\hline\hline}

%\newtheorem{theorem0}{Theorem}[section]
%\newtheorem{lemma0}{Lemma}[section]
%\newtheorem{remark0}{Remark}[section]
%\newtheorem{fact0}{Fact}[section]
%\newtheorem{example0}{Example}[section]
%\newtheorem{definition0}{Definition}[section]
%\newtheorem{corollary0}{Corollary}[section]
%\newtheorem{proposition0}{Proposition}[section]
%\newtheorem{algorithmY}{Algorithm}[section]

\newtheorem{theorem0}{Theorem}
\newtheorem{lemma0}{Lemma}
\newtheorem{remark0}{Remark}
\newtheorem{fact0}{Fact}
\newtheorem{example0}{Example}
\newtheorem{definition0}{Definition}
\newtheorem{corollary0}{Corollary}
\newtheorem{proposition0}{Proposition}
\newtheorem{algorithmY}{Algorithm}

\newenvironment{theorem}{\begin{theorem0} \mbox{} }{\end{theorem0}}
\newenvironment{lemma}{\begin{lemma0} \mbox{}}{\end{lemma0}}
\newenvironment{remark}{\begin{remark0} \mbox{}}{\end{remark0}}
\newenvironment{fact}{\begin{fact0} \mbox{}}{\end{fact0}}
\newenvironment{example}{\begin{example0} }{\end{example0}}
\newenvironment{definition}{\begin{definition0}
\mbox{}}{\end{definition0}}
\newenvironment{corollary}{\begin{corollary0} \mbox{}
}{\end{corollary0}}
\newenvironment{proposition}{\begin{proposition0}\mbox{}
}{\end{proposition0}}
\newenvironment{algorithm1}{\begin{algorithmY}\mbox{}
}{\end{algorithmY}}

\newcommand{\reals}{\mbox{\rm I\kern-.20em R}}
\newcommand{\sreals}{\mbox{\small \rm I\kern-.20em R}}
\newcommand{\mylinel}{\renewcommand{\baselinestretch}{1.8}\tiny\small}
\newcommand{\goto}{\rightarrow}
\newcommand{\expect}{\E}
\newcommand{\pp}[1]{\,\vskip2mm\noindent{\it #1.}~\ignorespaces}
\newcommand{\frot}{\frac{1}{2}}
%\newenvironment{proof}{\vspace*{-2mm}\noindent{\sc Proof}. \mbox{} \rm}
 %  {\hfill\fbox{}}
 

\usepackage{bm}
\usepackage{ upgreek }

\usepackage[]{mcode}
\usepackage{graphicx}
\usepackage{arydshln}
\begin{document}
% --------------------------------------------------------------
%                         Start here
% --------------------------------------------------------------
\title{\Huge \textbf{STAT W4640 }\\
\Large\underline{Assignment One}
}%replace X with the appropriate number
\author{ \Large\textbf{Yicheng Wang}}
\maketitle
\section*{Chapter 1: Exercise 7}
There are three boxes: Box A, Box B, Box C. The probability of three box has the big prize will be: $$Pr(A)=Pr(B)=Pr(C)=\frac{1}{3}$$ Without the loss of generality,If Box A was chosen.
Let the Host Open Box C.The probability of the host opening the Box C will be: 
If A has the big prize, the host could open Box C or B: $Pr(Open C|A)=\frac{1}{2}$.If  B has the big prize,the host has to open Box C, $Pr(Open C|B)=1$. If C has the big prize, the host could never open Box C $Pr(Open C|C)=0$.
$$Pr(Open C)=\frac{1}{3}*(\frac{1}{2}+1+0)=\frac{1}{2}$$ 
Therefore,  the probability A has the big prize and the host open Box C will be:$$Pr(A|Open C)=\frac{Pr(Open C|A)Pr(A)}{Pr(Open C)}=\frac{\frac{1}{2}\cdot \frac{1}{3}}{\frac{1}{2}}=\frac{1}{3}$$ 
The probability B has the big prize and the host open Box C will be:$$Pr(B|Open C)=\frac{Pr(Open C|B)Pr(B)}{Pr(Open C)}=\frac{1\cdot \frac{1}{3}}{\frac{1}{2}}=\frac{2}{3}$$. After all, we suggest change the choice to the other unchosen box.
%Box A has the big prize and Box B,and C have the lesser prize. Probability of getting three boxes will be $$Pr(A)=Pr(B)=Pr(C)=\frac{1}{3}$$. Without the loss of generality, we suppose the guy chose box B. And the host open Box C. Pr(A|C)= 
%%Set A as the event of getting the box with fabulous prize, set $B_1$ as the event of getting one of the box with lesser prize, and set $B_2$ as the event of getting one of the box with lesser prize. As we know, $$Pr(A)=Pr(B_1)=Pr(B_2)$$
%Without losing the generality, we suppose the contestant got the $B_1$ box.
%
\section*{Computational Problem}
The prior density will be:$$p(\theta)= \frac{\alpha-\alpha^d}{1-\alpha}\cdot\frac{\theta}{\alpha}+\frac{1-\alpha}{1-\alpha^d}\cdot\theta $$
The likelihood function will be $$p(y|\theta)\propto \theta^y(1-\theta)^{n-y}$$
Therefore the posterior density will be: 
\begin{align*}
p(\theta|y)\propto p(y|\theta)\cdot p(\theta) &=\frac{1-\alpha^{d-1}}{1-\alpha}\cdot\theta^{1+y}(1-\theta)^{n-y}+\frac{1-\alpha}{1-\alpha^d}\cdot\theta^{1+y}(1-\theta)^{n-y}\\
&=\frac{(1-\alpha)^2+(1-\alpha^d)(1-\alpha^{d-1})}{(1-\alpha)(1-\alpha^d)}\theta^{1+y}(1-\theta)^{n-y}
\end{align*}
\section*{Chapter 2: Exercise 19}
\subsection*{(a)}
\begin{align*}
&p(\theta) \propto\ \theta^{\alpha-1}\exp(-\beta \theta)\\
&p(y|\theta) \propto \theta \exp(\theta y)\\
&p(\theta|y)\propto p(\theta)\cdot p(y|\theta)=\theta^{\alpha}\exp[(y-\beta)\theta]
\end{align*}
Therefore, the posterior density will have gamma distribution with $\alpha+1$ and $\beta-y$ two parameters. The posterior density and the prior density are both gamma distributed. Thus, we have conjugate prior distribution.
\subsection*{(b)}

\subsection*{(c)}
\subsection*{(d)}

\section*{Chapter 2: Exercise 21}
\subsection*{(a)}
\subsection*{(b)}
\subsection*{(c)}

\newpage
\section*{Problem 8.3}

\newpage
\section*{Problem 9.3}

%\section*{Problem 5.12}
%\subsection*{Answer for 5.12 (i)}
%We defined in 5.4.7 that $$B_i(t)=\sum_{j=1}^d \int_0^t \frac{\sigma_{ij}(u)}{\sigma_i(u)}dW_j(u)$$
%$$\gamma_i(t)=\sum_{j=1}^d \frac{\sigma_{ij}(t)\theta_j(t)}{\sigma_i(t)}$$
%$$\tilde B_i(t)=\sum_{j=1}^d \int_0^t(\frac{\sigma_{ij}(t)\theta_j(t)}{\sigma_i(t)}dW_j(u)+\gamma_i(u)du)$$
%Hence, $$d\tilde B_i(t)=$$
%\subsection*{Answer for 5.12 (ii)}
%\subsection*{Answer for 5.12 (iii)}
%\subsection*{Answer for 5.12 (iv)}
%\subsection*{Answer for 5.12 (v)}
%
%
%\section*{Problem 8.3}
%\section*{Problem 9.3}
%
%\subsection*{Answer for 5.1 (i)}
%Based on 5.2.19, $f(x)=S(0)\exp[x]$.
%$$dX(t)=\sigma(t)dW(t)+[\alpha(t)-R(t)-\frac{1}{2}\sigma^2(t)]dt$$
%$$D(t)S(t)=f(X(t))=S(0)\exp[\int_0^t \sigma(s)dW(s)+\int_0^t(\alpha(s)-R(s)-\frac{1}{2}\sigma^2(s))ds]$$
%\begin{align*}
%d (D(t)S(t))&=df(X(t))
%=\frac{\partial f}{\partial x}(t,X(t))dX(t)+\frac{1}{2}\frac{\partial^2 f}{\partial x^2}(t,X(t))dX(t)dX(t)\\
%&=f(X(t)) dX(t)+\frac{1}{2}f(X(t)) dX(t)dX(t)\\
%&=f(X(t))[dX(t)+\frac{1}{2}dX(t)dX(t)]\\
%&=f(X(t))[dX(t)+\frac{1}{2}\sigma^2(t)dt]\\
%&=f(X(t))[\sigma(t)dW(t)+[\alpha(t)-R(t)-\frac{1}{2}\sigma^2(t)]dt+\frac{1}{2}\sigma^2(t)dt]\\
%&=D(t)S(t)[\sigma(t)dW(t)+[\alpha(t)-R(t)]dt]\\
%&=D(t)S(t)\sigma(t)[\Theta(t)dt+dW(t)]
%\end{align*}
%%\begin{align*}
%%\intertext{Focusing on the right hand side of the equation}
%%d(D(t)S(t))&=S(t)dD(t)+D(t)dS(t)+dD(t)dS(t)\\
%%&=-S(t)R(t)D(t)dt+D(t)[\alpha S(t)dt+\sigma(t)S(t)dW(t)]+[R(t)D(t)dt][\alpha S(t)dt+\sigma(t)S(t)dW(t)]\\
%%&=-S(t)R(t)D(t)dt+D(t)[\alpha S(t)dt+\sigma(t)S(t)dW(t)]\\
%%&=[\alpha -R(t)]D(t)S(t)dt+\sigma(t)S(t)dW(t)]
%%\end{align*}
%\subsection*{Answer for 5.1 (ii)}
%\begin{align*}
%\intertext{Focusing on the right hand side of the equation}
%d(D(t)S(t))&=S(t)dD(t)+D(t)dS(t)+dD(t)dS(t)\\
%&=-S(t)R(t)D(t)dt+D(t)[\alpha S(t)dt+\sigma(t)S(t)dW(t)]+[R(t)D(t)dt][\alpha S(t)dt+\sigma(t)S(t)dW(t)]\\
%&=-S(t)R(t)D(t)dt+D(t)[\alpha S(t)dt+\sigma(t)S(t)dW(t)]\\
%&=[\alpha -R(t)]D(t)S(t)dt+\sigma(t)D(t)S(t)dW(t)]\quad \text{Because $\Theta(t)=\frac{\alpha -R(t)}{\sigma(t)}$}\\
%&=D(t)S(t)\sigma(t)[\Theta(t)dt+dW(t)]
%\end{align*}
%
%\section*{Problem 5.2}
%$$Z(t)=\exp\Big\{-\int_0^t\Theta(u)dW(u)-\frac{1}{2}\int_0^t\theta^2(u)du\Big\}$$
%$$D(t)=\exp[-\int_0^tR(s)ds]$$
%Since $D(T)V(T)$ is martingale under $\tilde{\bfE} $ risk neutral measure, then 
%$$D(t)V(t)=\tilde{\bfE}[D(T)V(T)|\mathcal{F}(t)]=E[D(T)V(T)Z(T)|\mathcal{F}(t)]/Z(t)$$
%After we organize the both side of the equation, we can get
%$$D(t)V(t)Z(t)=E[D(T)V(T)Z(T)|\mathcal{F}(t)]\quad \qed$$
%\section*{Problem 5.3}
%\subsection*{Answer for 5.3 (i) }
%if $S(T)>K,(S(T)-K)^+=S(T)-K$
%\begin{align*}
%c_x(0,x)=\frac{\partial c(0,x)}{\partial x}=\tilde{\bfE}[\exp(-rT)\cdot\exp(\sigma \tilde{W}(T)+rT-\frac{1}{2}\sigma^2T)]=\tilde{\bfE}[\exp(\sigma \tilde{W}(T)-\frac{1}{2}\sigma^2T)]
%\end{align*}
%
%if $S(T)<K,(S(T)-K)^+=0$
%\begin{align*}
%c_x(0,x)=0
%\end{align*}
%Therefore, 
%$$c_x(0,x)=\tilde{\bfE}[\exp(\sigma \tilde{W}(T)-\frac{1}{2}\sigma^2T) \mathbbm{1}_{S(T)>K}]=\tilde{\bfE}[\bar{Z}(T)\cdot  \mathbbm{1}_{S(T)>K}]$$
%
%
%\subsection*{Answer for 5.3 (ii) }
%We set new $\bar{P}$ measure, let  $$\bar{Z}(T)=\frac{d\bar{P}}{d\tilde{P}}$$
%Therefore, $$c_x(0,x)=\tilde{\bfE}[\bar{Z}(T)\cdot  \mathbbm{1}_{S(T)>K}]=\bar{E}[\frac{\bar{Z}(T)}{\bar{Z}(T)}\cdot  \mathbbm{1}_{S(T)>K}]=\bar{E}[ \mathbbm{1}_{S(T)>K}]=  \bar{P}[({S(T)-K})^+]\quad \qed$$
%%We know that $$
%%c(x,t)=xN(d_{+})-Ke^{-r(T-t)}N(d_{-}).$$
%%Therefore, $$\frac{\partial c(x,t)}{\partial x}=c_{x}=N(d_{+})+xN^{'}(d_{+})\frac{\partial d_{+}}{\partial x}-Ke^{-r(T-t)}N{'}(d_{-})\frac{\partial d_{-}}{\partial x}$$
%%$$\frac{\partial d_{+}}{\partial x}=\frac{\partial d_{-}}{\partial x}=\frac{1}{\sigma\sqrt{T-t}x}$$
%%By using the conclusion in the part (i), we can get: $$\frac{\partial c(x,t)}{\partial x}=c_{x}=N(d_{+})+0\cdot \frac{1}{\sigma\sqrt{T-t}x}=N(d_{+})$$
%%\begin{align*}
%%\frac{\partial c(x,t)}{\partial t}=c(x)=
%%\end{align*}
%\begin{align*}
%\bar{W}(t)&=\tilde{W}(t)-\sigma t=\tilde{W}(t)-\int_0^t\sigma du\\
%d[\bar{W}(t)\bar{Z}(t)]&=d\bar{W}(t)\bar{Z}(t)+\bar{W}(t)d\bar{Z}(t)+f\bar{Z}(t)d\bar{W}(t)\\
%&=(-\bar{W}(t)\sigma+1)\bar{Z}(t)d\tilde{W}(t)\\
%\intertext{Since there is no $dt$ term, the process is a martingale under $\bar{P}$ } 
%\bar{E}[\bar{W}(t)|\mathcal{F}(s)]&=\frac{\tilde{E}[\bar{W}(t)\bar{Z}(t)|\mathcal{F}(s)]}{\bar{Z}(s)}=\frac{\bar{W}(s)\bar{Z}(s)}{\bar{Z}(s)}=\bar{W}(s) \qed
%\end{align*}
%\subsection*{Answer for 5.3 (iii) }
%\begin{align*}
%\bar{P}(S(T)>K)&=\bar{P}[x\Big(\exp(\sigma \tilde{W}(T)+(r-\frac{1}{2}\sigma^2)T)\Big)>K]=\bar{P}[x\Big(\exp(\sigma \bar{W}(T)+(r+\frac{1}{2}\sigma^2)T)\Big)>K]\\
%&=\bar{P}[ \bar{W}(T)>\frac{1}{\sigma}\Big(\ln\frac{ K}{x}-(r+\frac{1}{2}\sigma^2) T\Big)]\\
%&=\bar{P}[ -\frac{\bar{W}(T)}{\sqrt{T}}<\frac{1}{\sigma\sqrt{T}} \Big(\ln\frac{ x}{K}+(r+\frac{1}{2}\sigma^2) T\Big)]=N(d_{+}(T,x)) \qed
%\end{align*}
%\section*{Problem 5.4}
%\subsection*{Answer for 5.4 (i)}
%$$dS(t)=r(t)S(t)dt+\sigma(t)S(t)d\tilde{W}(t)$$
%\begin{align*}
%d\ln[S(t)]&=\frac{\partial \ln[S(t)]}{\partial x}dS(t)+\frac{1}{2}\frac{\partial^2 \ln[S(t)]}{\partial x^2}dS(t)dS(t)\\
%&=\frac{1}{S(t)}dS(t)-\frac{1}{2}(\frac{1}{S(t)})^2dS(t)dS(t)\\
%&=r(t)dt+\sigma(t)d\tilde{W}(t)-\frac{1}{2}\sigma^2 dt
%\end{align*}
%Therefore,$$\ln S(T)-\ln S(0)=\int_0^T\Big(r(t)-\frac{1}{2}\sigma^2\Big)dt+\int_0^T\sigma(t)d\tilde{W}(t)$$
%$$S(T)=S(0)\exp\Big[\int_0^T\Big(r(t)-\frac{1}{2}\sigma^2\Big)dt+\int_0^T\sigma(t)d\tilde{W}(t)\Big]$$
%
%\subsection*{Answer for 5.4 (ii)}
%According to what we have in the previous problem, 
%%Therefore,
%\begin{align*}
%c(0,S(0))&=\exp[-\int_0^Tr(t)dt]\tilde{E}[(S(T)-K)^{+}]\\
%\end{align*}
%We let $$T(R+\frac{1}{2}\Sigma^2)=\int_0^T r(t)dt+\frac{1}{2}\int_0^T\sigma^2(t)dt$$
%$$T(R-\frac{1}{2}\Sigma^2)=\int_0^T r(t)dt-\frac{1}{2}\int_0^T\sigma^2(t)dt$$
%
%\begin{align*}
%\tilde{E}[(S(T)-K)^{+}]&=\tilde{E}[\Big(S(0)\exp\Big[\int_0^T\Big(r(t)-\frac{1}{2}\sigma^2(t)\Big)dt+\int_0^T\sigma(t)d\tilde{W}(t)\Big]-K\Big)^+]\\
%&=\tilde{E}[\Big(S(0)\exp\Big[(R-\frac{1}{2}\Sigma^2)T+\Sigma \tilde{W}(t)\Big]-K\Big)^+]\\
%&=\exp[RT]BSM(T,S(0);K,R,\Sigma)\\
%\end{align*}
%Therefore,
%\begin{align*}
%c(0,S(0))&=\exp[-\int_0^Tr(t)dt]\tilde{E}[(S(T)-K)^{+}]\\
%&=\exp[-\int_0^Tr(t)dt]\exp[RT]BSM(T,S(0);K,R,\Sigma)\\
%&=BSM(T,S(0);K,R,\Sigma)\\
%&=BSM\Big(T,S(0);K,\frac{1}{T}\int_0^T r(t)dt,\sqrt{\frac{1}{T}\int_0^T\sigma^2(t)dt}\Big)
%\end{align*}
%
%After rebalancing, $$X(t+1)=\Delta(t+1)S(t+1)+\Gamma(t+1)M(t+1)$$
%By setting those two $X(t+1)$ equal, we can get$$ d\Delta(t)S(t+1)+d\Gamma(t)M(t+1)=0$$
%Then we can get $$S(t)d\Delta(t)+d\Delta(t)dS(t)+\Gamma(t)dM(t)+d\Gamma(t)dM(t)=0$$
\end{document}